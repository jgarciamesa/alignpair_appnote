\section{Results}
% prelim data comparison?

% by estimating $k_s$ and $k_a$ statistics.

% % Software comparison table
% \begin{table}[!ht]
% \centering
%     %\frame{
%     % \begin{table}[h!]
% \begin{adjustbox}{width=\columnwidth,center}
\definecolor{bestcolor}{RGB}{230,230,230}

\begingroup\centering
\begin{tabular}{r|ccccc}
      & \textbf{COATi} & \textbf{PRANK} & \textbf{MAFFT} & \textbf{Clustal$\Omega$} & \textbf{MACSE}\\
\hline
Avg alignment error ($d_{seq}$) & \cellcolor{bestcolor}0.00101 & 0.01010 & 0.00982 & 0.01582 & 0.00932\\
Perfect alignments & \cellcolor{bestcolor}2452 & 22 & 2175 & 1150 & 1580\\
Best alignments & \cellcolor{bestcolor}3624 & 155 & 2763 & 1609 & 2081\\
Imperfect alignments & \cellcolor{bestcolor}1136 & 3566 & 1413 & 2438 & 2008\\
% \hline
F1 score of positive selection & \cellcolor{bestcolor}90.8\pct & 80.5\pct & 73.5\pct & 61.3\pct & 70.6\pct\\
F1 score of negative selection & \cellcolor{bestcolor}99.1\pct & 98.0\pct & 97.2\pct & 96.0\pct & 97.4\pct
\end{tabular}
\par\endgroup
% \end{adjustbox}
% \end{table}

% comparison data Fall 2022 - 16k
%mecm        0.00100 6916662321
%macse       0.00932 470705850846
%mafft       0.00981 827284723824
%marginal    0.00101 295157751534
%clustalo    0.01582 33765322671
%prank       0.01009 5207890013
%fst         0.00109 586573083504
%dna         0.00109 485711623582
%ecm         0.00116 793745997768
%
% [1] "Perfect alignments"
%     mecm    macse    mafft   mcoati clustalo    prank    coati      dna    ecm
%     2438     1580     2175     2452     1150       22     2517     2577    2506
% [1] "Best alignments"
% [1] 3608     2081     2763     3624     1609      155     3602     3652    3639
% [1] "Imperfect alignments"
% [1] 1150     2008     1413     1136     2438      3566    1071     1011    1082
%
% [1] "Species gorilla with model mecm"
% [1] "ka root mean-squared error:0.00100926226681819"
% [1] "ks root mean-squared error:0.00236978661633702"
% [1] "Accuracy of + selection:0.915637860082305"
% [1] "Accuracy of - selection:0.991653094462541"
% [1] "Species gorilla with model macse"
% [1] "ka root mean-squared error:0.0327146249833387"
% [1] "ks root mean-squared error:0.0753999514529835"
% [1] "Accuracy of + selection:0.705882352941177"
% [1] "Accuracy of - selection:0.973769168684423"
% [1] "Species gorilla with model mafft"
% [1] "ka root mean-squared error:0.041389981600255"
% [1] "ks root mean-squared error:0.0528993120504005"
% [1] "Accuracy of + selection:0.735322425409047"
% [1] "Accuracy of - selection:0.971815107102593"
% [1] "Species gorilla with model mcoati"
% [1] "ka root mean-squared error:0.00108452715625288"
% [1] "ks root mean-squared error:0.00232522345414204"
% [1] "Accuracy of + selection:0.907975460122699"
% [1] "Accuracy of - selection:0.990833163577103"
% [1] "Species gorilla with model clustalo"
% [1] "ka root mean-squared error:0.0562479975129838"
% [1] "ks root mean-squared error:0.0790533430944015"
% [1] "Accuracy of + selection:0.612612612612613"
% [1] "Accuracy of - selection:0.960498111666837"
% [1] "Species gorilla with model prank"
% [1] "ka root mean-squared error:0.037825885981252"
% [1] "ks root mean-squared error:0.160220347970475"
% [1] "Accuracy of + selection:0.804503582395087"
% [1] "Accuracy of - selection:0.980464355119157"
% [1] "Species gorilla with model coati"
% [1] "ka root mean-squared error:0.00101421595739472"
% [1] "ks root mean-squared error:0.0024852941001945"
% [1] "Accuracy of + selection:0.919087136929461"
% [1] "Accuracy of - selection:0.99206672091131"
% [1] "Species gorilla with model dna"
% [1] "ka root mean-squared error:0.00106650180637591"
% [1] "ks root mean-squared error:0.00242713485387215"
% [1] "Accuracy of + selection:0.916752312435766"
% [1] "Accuracy of - selection:0.991754046625267"
% [1] "Species gorilla with model ecm"
% [1] "ka root mean-squared error:0.00104689678441889"
% [1] "ks root mean-squared error:0.0026750523913319"
% [1] "Accuracy of + selection:0.92713833157339"
% [1] "Accuracy of - selection:0.992994212610417"
% Warning message:
% In k_metric(species = ARGS[1], ARGS[2:length(ARGS)]) :
%   Number of sequences between ref and prank differs by 21
%}
% 	\caption{Accuracy of COATi, PRANK, MAFFT, Clustal$\Omega$, and MACSE,
%             on 2340 simulated sequence pairs. Perfect alignments have
%             ($d_{seq}=0$), best alignments have lowest $d_{seq}$, and imperfect
%             alignments have $d_{seq}>0$ when at least one aligner found a
%             perfect alignment. Best values are highlighted in blue.}
% 	\label{table:comp}
% \end{table}

Both models in COATi were significantly (p < $2.2$x$10^{-16}$) more accurate at
inferring simulated alignments compared to other aligners according to the
one-tailed Wilcoxon signed ranked test, with an average alignment error
($d_{seq}$) value of $1$x$10^{-3}$.
In addition, COATi produced more perfect alignments ($d_{seq}=0$), less imperfect
alignments, and more accurately retrieved events of positive selection
(fst model 91.9\%, marginal model 90.8\%) (Supplementary Table 1).
Compared to COATi, MACSE (allows frameshifts) and MAFFT (using DNA) had an
average alignment error an order of magnitude larger than COATi and a lower
accuracy retrieving events of positive selection at a rate of 81.5\%
and 85.8\% respectively.
PRANK (using codons) and Clustal$\Omega$ (using amino acid translations) had
an average alignment error two orders of magnitude larger than COATi and a
87.3\% and 69.1\% accuracy retrieving events of positive selection, respectively.
The accuracy retrieving events of negative selection was similar across all five
aligners (97.7\% $\pm$ 1.67\%).

% (Table \ref{table:comp}).


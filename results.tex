\section{Results}
% prelim data comparison?

% by estimating $k_s$ and $k_a$ statistics.

% % Software comparison table
% \begin{table}[!ht]
% \centering
%     %\frame{
%     % \begin{table}[h!]
% \begin{adjustbox}{width=\columnwidth,center}
\definecolor{bestcolor}{RGB}{230,230,230}

\begingroup\centering
\begin{tabular}{r|ccccc}
      & \textbf{COATi} & \textbf{MAFFT} & \textbf{PRANK\footnotesize{*}} & \textbf{MACSE} & \textbf{Clustal$\Omega$}\\
\hline
Method    & Trip-MG & DNA & Codon & DNA+AA & AA\\[2pt]
%\hline
Avg alignment error ($d_{seq}$) & \cellcolor{bestcolor}0.00214 & 0.01392 & 0.02001 & 0.01351 & 0.02691\\
Perfect alignments & \cellcolor{bestcolor}5722 & 5408 & 4706 & 2860 & 2937\\
Best alignments & \cellcolor{bestcolor}5152 & 4833 & 4748 & 3754 & 2595\\
Imperfect alignments & \cellcolor{bestcolor}1066 & 1380 & 2082 & 3928 & 3851\\
% \hline
F1 score of positive selection & \cellcolor{bestcolor}98.2\pct & 86.1\pct & 88.4\pct & 81.2\pct & 71.0\pct \\
F1 score of negative selection & \cellcolor{bestcolor}99.8\pct & 98.6\pct & 98.8\pct & 98.3\pct & 97.0\pct
\end{tabular}
\par\endgroup
% \end{adjustbox}
% \end{table}

%}
% 	\caption{Accuracy of COATi, PRANK, MAFFT, Clustal$\Omega$, and MACSE,
%             on 2340 simulated sequence pairs. Perfect alignments have
%             ($d_{seq}=0$), best alignments have lowest $d_{seq}$, and imperfect
%             alignments have $d_{seq}>0$ when at least one aligner found a
%             perfect alignment. Best values are highlighted in blue.}
% 	\label{table:comp}
% \end{table}

Both models in COATi were significantly (p < $2.2$x$10^{-16}$) more accurate at
inferring simulated alignments compared to other aligners according to the
one-tailed Wilcoxon signed ranked test, with an average alignment error
($d_{seq}$) value of $1$x$10^{-3}$.
In addition, COATi produced more perfect alignments ($d_{seq}=0$), less imperfect
alignments, and more accurately retrieved events of positive selection
(fst model 91.9\%, marginal model 90.8\%) (Supplementary Table 1).
Compared to COATi, MACSE (allows frameshifts) and MAFFT (using DNA) had an
average alignment error an order of magnitude larger than COATi and a lower
accuracy retrieving events of positive selection at a rate of 81.5\%
and 85.8\% respectively.
PRANK (using codons) and Clustal$\Omega$ (using amino acid translations) had
an average alignment error two orders of magnitude larger than COATi and a
87.3\% and 69.1\% accuracy retrieving events of positive selection, respectively.
The accuracy retrieving events of negative selection was similar across all five
aligners (97.7\% $\pm$ 1.67\%).

% (Table \ref{table:comp}).


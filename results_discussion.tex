\section*{Results and Discussion}
Using 16000 human genes and their gorilla orthologs from the ENSEMBL
database \citembe{ensembl_hubbard_2002}, we simulated a data set of pairwise
alignments with empirical gap patterns.
We used the data set to evaluate the accuracy of COATi and a suite of popular aligners spanning multiple alignment methods:
Clustal$\Omega$ v1.2.4 \citembe{clustal_omega_sievers_2011},
MACSE v2.06 \citembe{ranwez_macse_2011}, MAFFT v7.505
\citembe{katoh2013mafft}, and PRANK v.150803 \citembe{prank_loytynoja_2014}.

After downloading, we removed 2232 sequences longer than 6000 nucleotides, identified 6048 sequence pairs that contained gaps identified by at least one aligner, and 7719 ungapped sequence pairs.
We then randomly introduced gap patterns extracted from all five methods into the ungapped sequence pairs to generate the benchmark alignments.
Alignment accuracy was measured using the distance metric, $d_{seq}$
\citembe{metrics_blackburne_whelan_2011}, between simulated and inferred
alignments.
In addition, accuracy of positive and negative selection was calculated
using the $F_1$ score by estimating $k_s$ and $k_a$ statistics
\citembe{ka_ks_li_1993} (Supplementary Methods).
The $F_1$ score evaluates the accuracy of a model by assigning equal importance to precision and recall, and ranges between 0 and 1, with a score of 0 representing a perfect result.

% Software comparison table
\begin{table}[!ht]
\centering
% \begin{table}[h!]
% \begin{adjustbox}{width=\columnwidth,center}
\definecolor{bestcolor}{RGB}{230,230,230}

\begingroup\centering
\begin{tabular}{r|ccccc}
      & \textbf{COATi} & \textbf{PRANK} & \textbf{MAFFT} & \textbf{Clustal$\Omega$} & \textbf{MACSE}\\
\hline
Avg alignment error ($d_{seq}$) & \cellcolor{bestcolor}0.00101 & 0.01010 & 0.00982 & 0.01582 & 0.00932\\
Perfect alignments & \cellcolor{bestcolor}2452 & 22 & 2175 & 1150 & 1580\\
Best alignments & \cellcolor{bestcolor}3624 & 155 & 2763 & 1609 & 2081\\
Imperfect alignments & \cellcolor{bestcolor}1136 & 3566 & 1413 & 2438 & 2008\\
% \hline
F1 score of positive selection & \cellcolor{bestcolor}90.8\pct & 80.5\pct & 73.5\pct & 61.3\pct & 70.6\pct\\
F1 score of negative selection & \cellcolor{bestcolor}99.1\pct & 98.0\pct & 97.2\pct & 96.0\pct & 97.4\pct
\end{tabular}
\par\endgroup
% \end{adjustbox}
% \end{table}

% comparison data Fall 2022 - 16k
%mecm        0.00100 6916662321
%macse       0.00932 470705850846
%mafft       0.00981 827284723824
%marginal    0.00101 295157751534
%clustalo    0.01582 33765322671
%prank       0.01009 5207890013
%fst         0.00109 586573083504
%dna         0.00109 485711623582
%ecm         0.00116 793745997768
%
% [1] "Perfect alignments"
%     mecm    macse    mafft   mcoati clustalo    prank    coati      dna    ecm
%     2438     1580     2175     2452     1150       22     2517     2577    2506
% [1] "Best alignments"
% [1] 3608     2081     2763     3624     1609      155     3602     3652    3639
% [1] "Imperfect alignments"
% [1] 1150     2008     1413     1136     2438      3566    1071     1011    1082
%
% [1] "Species gorilla with model mecm"
% [1] "ka root mean-squared error:0.00100926226681819"
% [1] "ks root mean-squared error:0.00236978661633702"
% [1] "Accuracy of + selection:0.915637860082305"
% [1] "Accuracy of - selection:0.991653094462541"
% [1] "Species gorilla with model macse"
% [1] "ka root mean-squared error:0.0327146249833387"
% [1] "ks root mean-squared error:0.0753999514529835"
% [1] "Accuracy of + selection:0.705882352941177"
% [1] "Accuracy of - selection:0.973769168684423"
% [1] "Species gorilla with model mafft"
% [1] "ka root mean-squared error:0.041389981600255"
% [1] "ks root mean-squared error:0.0528993120504005"
% [1] "Accuracy of + selection:0.735322425409047"
% [1] "Accuracy of - selection:0.971815107102593"
% [1] "Species gorilla with model mcoati"
% [1] "ka root mean-squared error:0.00108452715625288"
% [1] "ks root mean-squared error:0.00232522345414204"
% [1] "Accuracy of + selection:0.907975460122699"
% [1] "Accuracy of - selection:0.990833163577103"
% [1] "Species gorilla with model clustalo"
% [1] "ka root mean-squared error:0.0562479975129838"
% [1] "ks root mean-squared error:0.0790533430944015"
% [1] "Accuracy of + selection:0.612612612612613"
% [1] "Accuracy of - selection:0.960498111666837"
% [1] "Species gorilla with model prank"
% [1] "ka root mean-squared error:0.037825885981252"
% [1] "ks root mean-squared error:0.160220347970475"
% [1] "Accuracy of + selection:0.804503582395087"
% [1] "Accuracy of - selection:0.980464355119157"
% [1] "Species gorilla with model coati"
% [1] "ka root mean-squared error:0.00101421595739472"
% [1] "ks root mean-squared error:0.0024852941001945"
% [1] "Accuracy of + selection:0.919087136929461"
% [1] "Accuracy of - selection:0.99206672091131"
% [1] "Species gorilla with model dna"
% [1] "ka root mean-squared error:0.00106650180637591"
% [1] "ks root mean-squared error:0.00242713485387215"
% [1] "Accuracy of + selection:0.916752312435766"
% [1] "Accuracy of - selection:0.991754046625267"
% [1] "Species gorilla with model ecm"
% [1] "ka root mean-squared error:0.00104689678441889"
% [1] "ks root mean-squared error:0.0026750523913319"
% [1] "Accuracy of + selection:0.92713833157339"
% [1] "Accuracy of - selection:0.992994212610417"
% Warning message:
% In k_metric(species = ARGS[1], ARGS[2:length(ARGS)]) :
%   Number of sequences between ref and prank differs by 21

 \vspace{1mm}
 \footnotesize{* PRANK produced 50 empty alignments, calculations are based on 7869 alignments.}
 \caption{COATi generates better alignments than other alignment algorithms. Results of COATi, PRANK, MAFFT, Clustal$\Omega$, and MACSE aligning 7719 empirically simulated sequence pairs. Perfect alignments have the same score as the true alignment, best alignments have the lowest $d_{seq}$, and imperfect alignments have a different score than the true alignment when at least one method found a perfect alignment.}
 \label{table:comp}
\end{table}

COATi, using the codon-triplet-mg model, obtained better results compared to a wide variety of alignment strategies.
It was significantly more accurate (lower $d_{seq}$) at inferring simulated alignments compared to other methods; all p-values were less than $2.1 \cdot 10^{-79}$ according to the one-tailed, paired Wilcoxon signed-rank tests (Supplementary Materials Figure 1).
%
In addition, COATi produced more perfect alignments, less imperfect alignments, and more accurately retrieved events of positive selection (Table \ref{table:comp}).

Clustal$\Omega$ generated alignments via amino acid translations and obtained the highest average alignment error while having difficulties retrieving positive selection.
MACSE used a DNA-AA hybrid model, allowing frameshifts, and obtained similar results to MAFFT using a DNA model.
PRANK, using a codon model, had a similar average alignment error between MACSE/MAFFT and Clustal$\Omega$ but was unable to generate alignments for some sequence pairs.

Despite human and gorilla sequences having a relatively short evolutionary distance, COATi showed a biologically significant improvement over other methods, with an average alignment error nine-fold smaller than the next best method.
COATi is an FST-based application that can calculate the optimal alignment
between a pair of sequences in the presence of artifacts using a statistical
model.
Using COATI will allow researchers to analyze more data with higher accuracy and facilitate the study of important biological processes that shape genomic data.

Future work include extending the indel FST to combine a 3-mer gap model with a frameshift parameter and weighing each indel phase differently to reflect known selection on indel phases \citembe{zhu2022profiling}.
We also plan on comparing the marginal and triplet models to evaluate the implications of the marginalization.
